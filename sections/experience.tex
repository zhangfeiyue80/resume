\documentclass[../main.tex]{subfiles}%
%
\SECT{Experience}{1.9em}
{高伟达软件股份有限公司}
{中国邮政储蓄银行客户信息平台系统开发项目组}
%
{\begin{adjustwidth}{1.2in}{0in}
\begin{flushleft}\setstretch{1.25}
	%
	\SECTCONT{客户信息平台系统底层数据库从 oracle 转为 postgres, 负责 memcache 逻辑操作日志文件到 postgres 的持久化存储的同步服务编写, 在原有架构上的底层迁移; 实现 postgres 集群节点的数据分布式存储访问}
	\SECTCONT{客户信息平台系统配合信审系统查询客户风险指标信息,在日终批量架构上,新增了一支 xml 文件导入任务,在基于 tuxedo 的交易架构上新增了一支数据查询交易}
	%
	\SECTC{天融信网络安全}
		{wireshark 报文分析}
		{ssh,tftp,radius,smb,nfs 协议维护}
	\SECTCONT{网审文件还原程序clean\_file 添加文件还原大小统计功能}

	%
	\SECTC{北京南天软件有限公司}
		{运维开发邮储业务系统 (新网点核算,储蓄网点报表, 亦庄储蓄逻辑集中历史)}
		{参与了新增系统业务功能 (linux pro*c),优化系统性能,且是这些项目的主干力量}
	%
	\SECTC{Python/C/C++}
		{github小功能}
		{链表, n叉数 \HREF{https://github.com/zhangfeiyue80/BaseTraning}{BaseTraining}; librosa库音频去底噪, matplotlib库画时序图,频谱图\HREF{https://github.com/zhangfeiyue80/PyFilterWav}{PyFilterWav}}
	%
\end{flushleft}
\end{adjustwidth}}
