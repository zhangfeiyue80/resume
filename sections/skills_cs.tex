\documentclass[../main.tex]{subfiles}%
%
\SECT{Skills (CS)}{1.4em}
{Programming}
{C/C++,Python,Java $\vert$ Gcc+make+shell $\vert$ Vim+tmux 
			\HREF{https://github.com/zhangfeiyue80/myyum/blob/master/.vimrc} {.vimrc}
			\HREF{https://github.com/zhangfeiyue80/myyum/blob/master/.tmux.conf} {.tmux.conf}
			\HREF{https://github.com/zhangfeiyue80/myyum/blob/master/.profile} {.profile}}
%
{\begin{adjustwidth}{1.2in}{0in}
\begin{flushleft}\setstretch{1.25}
	%
	\SECTCONT{X86 汇编, 动/静态链接机制,  函数的汇编实现机制, ELF(executable and linkable format)文件格式}
	\SECTCONT{了解基本数据结构(链表,队列, 栈, 树), 算法(排序查找算法) }
	\SECTCONT{了解多进程, 多线程编程(互斥锁) }
	\SECTCONT{了解IPC(Inter-process communication)-管道/消息队列/信号量/共享内存/套接字}
	\SECTCONT{基于memcache/postgres分布式应用实现}
	\SECTCONT{基于tuxedo中间件C/S应用程序开发}
	\SECTCONT{面向对象编程,了解设计模式(生产者-消费者)}
	\SECTCONT{基于已有系统框架进行功能开发}
	%
	\SECTC{Network Development }
		{IP/TCP/UDP | HTTP, tcp连接建立释放,三次握手四次挥手}
		{了解IP/TCP协议, SMB, TFTP, NFS, HTTP应用层协议}
	\SECTCONT{SOCKET网络编程, HTML, PYTHON, SQL}
	%
	\SECTC{DBMS \& System Administration} 
		{Ubuntu, CentOS/RHEL, Debian} 
		{Bash script automation \& 功能复用 (shell lib 库文件),bash script 增强交互 (dialog),vim}
	\SECTCONT{oracle SQL, oracle procedure, oracle 性能优化 (建索引,建分区), postgres}
	%
\end{flushleft}
\end{adjustwidth}}
